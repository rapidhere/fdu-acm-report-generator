\documentclass[a4paper, 11pt, nofonts, nocap, fancyhdr]{ctexart}

\usepackage[margin=60pt]{geometry}

\setCJKmainfont[BoldFont={FZHTK--GBK1-0}, ItalicFont={FZHTK--GBK1-0}]{FZSSK--GBK1-0}
\setCJKsansfont{FZHTK--GBK1-0}
\setCJKmonofont{FZSSK--GBK1-0}

\CTEXoptions[today=small]

\pagestyle{plain}

% \fancyhead[L]{\small{team name}}
% \fancyhead[C]{\small{FSTC 2014 - 05 - 训练报告}}
% \fancyhead[R]{\small{2014年8月2日}}

\renewcommand{\thesubsubsection}{Problem \Alph{subsubsection}.}
\newcommand{\problem}[1]{\subsubsection{#1}}

\title{Fudan ACM-ICPC Summer Training Camp 2014\\第1场训练报告}
\author{team 7}
\date{\today}

\begin{document}

\maketitle

\section{概况}

我是概览我是概览我是概览

概览概览概览

还是概览


\section{训练过程}

我是比赛过程

比赛过程


\section{解题报告}

\problem{A Problem}

\begin{description}
\item[负责] dq,htc
\item[情况] 赛后通过
\end{description}

这是A题的题解

\problem{B Problem}

\begin{description}
\item[负责] dq,htc
\item[情况] 赛后通过
\end{description}

这是B题的题解

\problem{C Problem}

\begin{description}
\item[负责] dq,htc,tdy
\item[情况] 比赛时通过
\end{description}

这是C题的题解



\section{总结}

我是总结我是总结我是总结

总结


\end{document}
